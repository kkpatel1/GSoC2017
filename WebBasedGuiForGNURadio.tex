\documentclass[a4paper, 11pt]{article}

\usepackage{graphicx}

\title{GSoC 2017: A HTML-based GUI for GNU Radio}
\author{Kartik Patel}
\begin{document}
\maketitle

\section{Introduction}
HTML based GUI are becoming more famous(used to, widespread) because of easy to use and easy to view outputs. Currently GNU Radio works on a standalone system (many times connected to a hardware). The output of GNU Radio is displayed using Python's QT GUI framework which allows to view the results only on the connected system. In addition to that, various input widgets of QT framework are used to interact with the ongoing simulations.

In this project, a HTML based GUI is proposed for GNU Radio. The primary focus of the project will be an output mechanism which will enable the output viewing through the HTML document. In addition to the output mechanism, interactive HTML inputs will be used to iteract with the ongoing simulations.

\subsection{Major features of the project}
\begin{enumerate}
\item Alternative output mechanism other than Python QT framework
\item Allow real-time interaction with the program remotely
\item Simultaneous real-time multi-user interaction with the program
\end{enumerate}

\section{Proposed mechanism and comparison with \texttt{gr-qtgui}}
At present, GNU Radio plots various figures in dialog box based on QT Framework. Similarly, the proposed module will show various figures through HTML page served from the system running the simulations. In particular, we will use Bokeh library to setup the server, sessions, documents and plots. Figure 1 provides comparison of proposed mechanism with the current structure.

\begin{figure}
\includegraphics[scale=0.5]{WebGui/fundamental.png}
\caption{Figure 1: Comparison of `gr-qtgui` with `gr-webgui`}
\end{figure}

On the client side, a web browser requests a page from server:port/?session-id=sessioni$\_$id. The server sends the Document object corresponding to the session identified by sessioni$\_$id. Since, there is only one Document and session instance on server, all web browsers receives same Document instance. Hence, changing the parameters or plot configurations will ensure the change in all clients viewing the document.

\section{Suggested features}
A summary of suggested features are as follows:

\begin{enumerate}
\item Following plots will be implemented for the web based GUI. Most plots are directly available in Bokeh library. Others can be developed by providing the formatted input values to existing plots in Bokeh library.
\begin{itemize}
\item Constellation Display
\item Histogram Display
\item BER Sink
\item TimeRaster Display
\item Frequency Display
\item Time Display
\item Waterfall Display
\end{itemize}

\item Following input widgets will be implemented for the web based GUI. All these widgets are already available in Bokeh library.
\begin{itemize}
\item Checkbox
\item Chooser (Drop-down menu in HTML)
\item TextBox
\item Label (Non-editable textbox)
\item Push Button
\item Range Slider
\item Slider
\item Tab Panes
\end{itemize}

\item Add GRC blocks for the Sinks
\begin{itemize}
\item Add option in Options block and define building of top$\_$block.py
\item Add GRC blocks for each sinks and input widgets mentioned above
\end{itemize}
\end{enumerate}

Finally, an OOT module is to be developed. Upon changing the Options block in GRC, web based GUI will be enabled and hence, a server process will be initiated. Opening the URL from web browser will show the expected plots and input widgets.

\section{Proposed work during GSoC 2017}
The miniature version of whole implementation is available here. The file time$\_$sink$\_$f.py implements basic sink for floating point input and the file top$\_$block.py is an example on how the sinks will be used. You can review the output here.

Proposed flow of the OOT module gr-webgui is explained below.

Need to finalize it as soon as possible.

\section{Personal background and previous experience}
I am final year undergraduate student at Department of Electronics and Communication Engineering, Indian Institute of Technology Roorkee. I will be joining $\_\_\_\_$ for Ph.D. in Electrical and Computer Engineering with majors in Electrical Engineering. My area of interests revolve around Communication systems and I have developed web and software development as my hobby. My major motivation to apply to GSoC 2017 is to get involved in GNU Radio development. Since, registration for my graduate studies are generally in beginning of September, I will be mostly available for the duration of May-August except 3-4 days sometimes in June or July to complete my visa process.

My experience in programming and in particular open-source development is as follows:
\begin{itemize}
\item \textbf{Implementation of Bluetooth Low Energy module in NS3} [Link] - Designed and implemented Bluetooth Low Energy protocol stack in NS3. Initially developed the basic idea here and then implemented whole module within 4 weeks.
\item \textbf{Chief Technical Lead, Information Management Group (IMG), IIT Roorkee} - IMG develops and maintains the IIT Roorkee Intranet \& Internet systems. We manage the Institute website, Content Management System, Placement Portal and many other applications for the institute. I am responsible for all aspects of the backend stack including performance management and security. I initiated changes in development cycle to optimize the resource usage of servers and reduce the frequency of communication between servers and databases.

\item In the beginning of 2017, I attempted to contribute to GNU Radio in order to get familiarize with organization and codebase. Following are my attempts to contribute to the code.
\begin{itemize}
\item Created the pull request with the feature to Duplicate flowgraph and Save a Copy (PR: \#1188).
\item Solved issue \#1124
\item Solved issue \#1192
\end{itemize}

\end{itemize}


\section{Secret code for 2017}
Cyberspectrum is the best spectrum.


\section{Conclusion}
An overview of the module for web based GUI for GNU Radio is given in the previous sections. With example of a small scale implementation, the flow of development is explained. The project is devided into proper timeline so that mentor and community can track the progress of the project.

\end{document}